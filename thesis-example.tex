% -*- coding:utf-8 mod:LaTeX -*-
% !TeX spellcheck = de_DE
%Neue deutsche Trennmuster
%Siehe http://www.ctan.org/pkg/dehyph-exptl und http://projekte.dante.de/Trennmuster/WebHome
%Nur für pdflatex, nicht für lualatex
% \RequirePackage[ngerman=ngerman-x-latest]{hyphsubst}
%Warns about outdated packages and missing caption declarations
%See https://www.ctan.org/pkg/nag
\RequirePackage[l2tabu, orthodox]{nag}
% TODO: b5?
\documentclass[paper=a5,twoside,fontsize=10pt, DIV=calc, headings=small, bibliography=totoc, listof=totoc]{scrbook}
\usepackage[utf8]{inputenc}

\input{shared/template}
\usepackage{shared/titlepage}

% binding correction
\setlength{\evensidemargin}{-24pt}
\setlength{\oddsidemargin}{-24pt}

%\input{commands}

\bibliography{bibliography}

% \hyphenation{
% Spe-zi-fi-ka-tion
% In-te-gra-tion
% An-for-de-rung An-for-de-run-gen
% Be-nut-zer-ober-flä-che
% Mes-sung-en
% aus-zu-tau-schen
% Lauf-zeit-in-for-ma-tionen
% % Dürfen nicht getrennt werden
% AROMA TOSCA BPMN OASIS OMG DMTF IT DevOps
% }

\begin{document}
\iftex4ht
\Configure{graphics*}  % After begin document. Package tex4ht
  {pdf}
  {\Link[\csname Gin@base\endcsname .png]{}{}% link the png
  \Picture[pict]{\csname Gin@base\endcsname .png width:80\% align=center border=0}%  Show the png
  \EndLink
  }

\Css{body {text-align:justify;}}

%all LaTeX formulas as pictures
\Configure{$}{\PicMath}{\EndPicMath}{}
%$ % <- syntax highlighting fix for emacs
\fi

%Tipp von http://goemonx.blogspot.de/2012/01/pdflatex-ligaturen-und-copynpaste.html
%siehe auch http://tex.stackexchange.com/questions/4397/make-ligatures-in-linux-libertine-copyable-and-searchable
%
%ONLY WORKS ON MiKTeX
%On other systems, download glyphtounicode.tex from http://pdftex.sarovar.org/misc/
%
\input glyphtounicode.tex
\pdfgentounicode=1

% Die Seitennummerierung erfolgt durchlaufend ab der Titelseite. Also keine
% Spielereien mit römischen Ziffern usw. - Die ISO 7144 schreibt das sogar für
% wissenschaftliche Werke vor.
% Von Promitionsordnung verlangt!
% Deshalb ist \frontmatter DEAKTIVIERT
%\frontmatter
\title{Thesis Title}
%\author{\texorpdfstring{\href{http://www.example.org/}{Author Name}}{Author Name}}
\author{Marcel Breyer}
\date{\today}
\keywords{TODO}
\firstexaminer{Prof.~Dr.~Dirk Pflüger}
\secondexaminer{Prof.~Dr.~???}
\dateofexamination{unbekannt}
\placeofbirth{Waiblingen}
\faculty{Fakultät für Informatik, Elektrotechnik und Informationstechnik}
\department{Institut für Parallel und Verteilte Systeme}

\maketitle

% This is necessary if you have more than 9 sections/subsections/subsubsections
% \makeatletter
% \renewcommand\l@section{\@dottedtocline{1}{1.5em}{3em}}
% \renewcommand\l@subsection{\@dottedtocline{2}{1.5em}{4.3em}}
% \renewcommand\l@subsubsection{\@dottedtocline{3}{1.5em}{5.6em}}
% \makeatother

\tableofcontents
\clearpage

\pagestyle{justpagenums}
% \input{abstract}

%\mainmatter
\pagestyle{scrheadings}

%% --- Zusammenfassung ---------------------------------------------
\input{content/00_Zusammenfassung}

%% --- Abstract Page---------------------------------------------
\input{content/00_abstract}

%% --- Acknowledgements page------------------------------------
\input{content/01_acknowledgements}

\pagestyle{scrheadings}

%% --- CONTENT ---
\input{content/000_dummy_content}

\printbibliography
\noindent
All URLs were last checked 2025.01.01.
% Alle URLs wurden zuletzt am 17.01.2014 gepr\"uft.

\clearpage
\listoffigures
\listoftables

\chapter*{List of Definitions}
\addcontentsline{toc}{chapter}{List of Definitions}
\listtheorems{definition}

\appendix

% 'Anhang' ins Inhaltsverzeichnis
%\phantomsection
%\addcontentsline{toc}{part}{Anhang}

%% --- Appendix ---
\chapter{Appendix}

\begin{figure}
Test
\caption{Beispielabbildung im Anhang}
\end{figure}

%\input{content/Z-Anhang}

\IfDefined{printindex}{\printindex}
\IfDefined{printnomenclature}{\printnomenclature}

\end{document}

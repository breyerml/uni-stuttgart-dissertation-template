\chapter{Test}
\label{chap:test}

\Citet{WSPA} beschreiben die Umsetzung einer serviceorientierten Architektur mittels Web-Services, während \citet{zMR2005} BPMN um den Aspekt des Risikomanagements erweitern.

Das Potenzmengensymbol $\powerset$ ist auch korrekt und kein kleines Weierstraß-p ($\wp$).

\textmarker{Markierter Text.}
\modified{Manuelle Markierung für Text, der seit der letzten Version geändert wurde.}

%provided indirectly by pdfcomment.sty (soulpos).
\hl{In Gelb hervorgehoben.}

Checkmark: \dingcheck. Crossmark: \dingcross.

Referencetest: \Cref{ssec:example}, \cref{fig:Abbildung} und \cref{alg:example}.

\begin{figure}
\missingfigure{}
\caption{Abbildung}
\label{fig:Abbildung}
\end{figure}

\begin{landscape}
\begin{figure}
\missingfigure{}
\caption{Gedrehte Abbildung}
\label{fig:AbbildungGedreht}
\end{figure}
\end{landscape}

\begin{algorithm}
\caption{$algo$}
\label{alg:example}
\begin{algorithmic}[1]
\State $a \gets 0$
\State State 2\label{alg1:state2}
\end{algorithmic}
\end{algorithm}

\section{Definitionen}
\begin{definition}[Title]
\label{def:def1}
Definition Text
\end{definition}

\Cref{def:def1} zeigt \ldots


\section{Aufzählungen}
\begin{enumerate}[label=\alph*)]
\item a
\item b
\item c
\item d
\end{enumerate}

Equivalent to paralist's inparaenum:
\begin{enumerate*}[label=\alph*)]
\item a
\item b
\item c
\item d
\end{enumerate*}

\begin{description}
\item[first] Erstens
\item[second] Zweitens
\item[third] Drittens
\end{description}

\begin{description}
\item[\texttt{first}] Erstens
\item[\texttt{second}] Zweitens
\item[\texttt{third}] Drittens
\end{description}

%works only if package enumitem is loaded
\begin{description}[font=\ttfamily]
\item[first] Erstens
\item[second] Zweitens
\item[third] Drittens
\end{description}

\begin{description}[style=unboxed]
\item[first label with a long description text breaking over one line. Enabled by enumitem package] Erstens
\item[second] Zweitens
\item[third] Drittens
\end{description}

\begin{Description}
\item[first label with a long description text breaking over one line. Defined in template.tex] Erstens
\item[second] Zweitens
\item[third] Drittens
\end{Description}

\begin{itemize}
\item Erstens
\item Zweitens
\item Drittens
\end{itemize}

Optionaler Parameter ändert den Marker, der vorangestellt ist. Siehe \url{http://www.weinelt.de/latex/item.html}.
\begin{itemize}
\item[A] Erstens
\item[B] Zweitens
\item[C] Drittens
\end{itemize}

Falsche Benutzung des optionalen Parameters wie folgt:
\begin{itemize}
\item[first] Erstens
\item[second] Zweitens
\item[third] Drittens
\end{itemize}
Dabei ist zu beachten, dass es sich bei Einbindung von \texttt{enumitem} anders verhält als bei \texttt{paralist}.


\section{fquote}

\begin{fquote}[T.\ Informatiker]
Bis nächsten Freitag ist das Programm fertig.
\end{fquote}

\begin{gfquote}{T.\ Informatiker}
Bis nächsten Freitag ist das Programm fertig.
\end{gfquote}

\todo{Hier muss noch kräftig Text produziert werden}

\section{Varioref Demonstration}
Siehe \vref{chap:test}.

\section{Algorithmen}
\begin{algorithm}
\caption{Algorithmus 2}
\label{alg:example2}
\begin{algorithmic}[1]
\State $a \gets 0$
\State State 2\label{alg2:state2}
\end{algorithmic}
\end{algorithm}

\Cref{alg:example} hat bereits einen Algorithmus gezeigt.
Test der Zeilenreferenzierung: Zeile~\ref{alg1:state2} (\cref{alg:example}) und Zeile~\ref{alg2:state2} (\cref{alg:example2}).

\section{Listing Demonstration}
Minted ist das beste, aber funktioniert nur, wenn \href{http://pygments.org/download/}{pygments} installiert ist und pdflatex mit \texttt{-shell-escape} ausgeführt wird.

\iffalse
\begin{listing}
\begin{minted}[linenos=true]{xml}
<process name="reisebuero">
</process>
\end{minted}
\caption{Beispielprozess}
\label{lst:beispielprozess}
\end{listing}
\fi

Alter Stil mittels des Listings-Pakets ist in \cref{lst:beispielprozess} gezeigt.
%Listing-Umgebung wurde durch \newfloat{Listing} definiert
\begin{lstlisting}[float,caption={Beispielprozess},label={lst:beispielprozess}]
<process name="reisebuero">
</process>
\end{lstlisting}

\section{Demonstration von refenums}
\label{sec:method}
%See http://mirror.ctan.org/macros/latex/contrib/refenums/demo.tex for a full usage guide
%Setup ``Steps'' not having a print form
\setupRefEnums[~]{Schritt}{ONLYSHORT}

\defRefEnum[subsection]{Schritt}{Anforderungsanalyse}{rqa}
\label{sec:rqa}
\blindtext

\defRefEnum[subsection]{Schritt}{Spezifikation}{spec}
\blindtext

\subsection{Zusammenfassung}
In \cref{sec:method} wurde zuerst \refEnumFull{Schritt}{rqa} vorgestellt.
Anschließend wurde \refEnumFull{Schritt}{spec} beschrieben.

\section{Demonstration für Kommentare}
Das ist ein Text.
\change{FL1: Text angepasst}{Geänderter Text}.
Hier nur ein Kommentar\comment{Kommentar}.

Alternativ: \modified{Nur geändert, ohne Rückverweis auf Korrekturkommentar}.


\section{Section}

\section{Section}
\blindtext

\section{Section}

\section{Section}

\subsection{Subsection}
\label{ssec:example}
\blindtext

\subsection{Subsection}

%\input{content/chapter1}
\setchapterfooter{Kleiner Fuß!}
\chapter{Transformation of WS-BPEL 2.0 into Executable Workflow Networks}
\label{chap:bpel-ewfn-transformation}

%Um die Fußzeile zu demonstrieren
\blindtext[3]

\resetchapterfooter